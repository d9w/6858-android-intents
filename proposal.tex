\documentclass[12pt,a4paper]{article} \usepackage{verbatim}
\usepackage{listings} \usepackage{hyperref} \usepackage{url}

\begin{comment} 
Write a document describing the design and implementation of your
project, and turn it in along with your project's code by the final
deadline.  
\end{comment}

\begin{document} \bibliographystyle{plain}

\flushleft\textbf{Exploiting common \texttt{Intent} vulnerabilities in Android %
applications}\\ \today\\

\paragraph{Members} ~\\

Kelly Casteel, \texttt{kcasteel}\\ Owen Derby, \texttt{oderby}\\ Dennis Wilson,
\texttt{dennisw}\\

\paragraph{Problem} ~\\ The Android framework allows apps and components within
apps to communicate with one another by passing messages, called
\href{https://developer.android.com/reference/android/content/Intent.html}{Intents},
which specify both a procedure to call and the arguments to use. Applications
must declare in a static manifest file which intents each component services, as
well as both application and component level permissions. While the security
vulnerabilities in outgoing Intents has been well studied
\cite{chin_analyzing_2011} and developer tools exist to limit potentially
insecure Intents, little has been done to address malicious incoming Intents.
Exploits of this nature have been discovered in firmware of various Android
phones \cite{grace_systematic_2012}, but exploits in third-party applications
are not well studied. Application developers must make sure their manifest file
has been properly configured to only accept desired Intents, which can limit
usability. We believe that developers will trust Intent input by default,
allowing malicious input to potentially crash or abuse the application.

\paragraph{Static analysis} ~\\

\paragraph{Common app vulnerabilites} -\\

\paragraph{Exploit} -\\

\paragraph{Conclusion} -\\

\bibliography{writeup} \end{document}
