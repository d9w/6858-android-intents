\documentclass[12pt,a4paper,draft]{article}
\usepackage{verbatim}
\usepackage{listings}
\addtolength{\voffset}{-60pt}

\begin{comment}
Project proposal. Discuss your proposed idea with course staff over the next
week, before the proposal deadline, to flesh out the exact problem you will be
addressing, how you will go about doing it, and what tools you might need in the
process. By the proposal deadline, you must submit a one-to-two-page proposal
describing: your group members list, the problem you want to address, how you
plan to address it, and what are you proposing to specifically design and
implement.
\end{comment}

\begin{document}
\flushleft\textbf{Exploiting common \texttt{Intent} vulnerabilities in Android %
applications}\\
\today\\

\paragraph{Members} ~\\

Kelly Casteel, \texttt{kcasteel}\\
Owen Derby, \texttt{oderby}\\
Dennis Wilson, \texttt{dennisw}\\

\paragraph{Problem} ~\\
The Android framework allows apps and components within apps to communicate with
one another by broadcasting messages to BroadcastReceivers \begin{comment} link
to documentation \end{comment} with messages contained in Intents. By default,
any app can communicate with a BroadcastReceiver, although they can be
configured to only accept messages from apps with appropriate privileges or from
other components of the same app.

\paragraph{Proposal} ~\\
We plan to build a static analysis tool that will look for vulnerabilities in
Android apps that result from bad intent processing. This analysis tool will
look at BroadcastReceivers that are not protected by system or signature
privileges and trace the input from those intents. If the input is used to
access sensitive data or priviliges without being sanitized, it will be marked
as a potential vulnerability. We will test our tool on apps found
\begin{comment}google code site\end{comment}. If we find an app with a bug that
will allow us to do something interesting, we will also create an app that
demonstrates the potential exploits.

\end{document}

