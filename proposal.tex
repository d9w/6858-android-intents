\documentclass[12pt,a4paper]{article}
\usepackage{verbatim}
\usepackage{listings}
\usepackage{hyperref}

\begin{comment}
Project proposal. Discuss your proposed idea with course staff over the next
week, before the proposal deadline, to flesh out the exact problem you will be
addressing, how you will go about doing it, and what tools you might need in the
process. By the proposal deadline, you must submit a one-to-two-page proposal
describing: your group members list, the problem you want to address, how you
plan to address it, and what are you proposing to specifically design and
implement.
\end{comment}

\begin{document}
\bibliographystyle{plain}

\flushleft\textbf{Exploiting common \texttt{Intent} vulnerabilities in Android %
applications}\\
\today\\

\paragraph{Members} ~\\

Kelly Casteel, \texttt{kcasteel}\\
Owen Derby, \texttt{oderby}\\
Dennis Wilson, \texttt{dennisw}\\

\paragraph{Problem} ~\\
The Android framework allows apps and components within apps to communicate with
one another by passing messages, called
\href{https://developer.android.com/reference/android/content/Intent.html}{Intents},
which specify both a procedure to call and the arguments to use. Applications
must declare in a static manifest file which intents each component services, as
well as both application and component level permissions. While the security
vulnerabilities in outgoing Intents has been well studied
\cite{chin_analyzing_2011} and developer tools exist to limit potentially
insecure Intents, little has been done to address malicious incoming
Intents. Exploits of this nature have been discovered in firmware of various
Android phones \cite{grace_systematic_2012}, but exploits in third-party
applications are not well studied. Application developers must make sure their
manifest file has been properly configured to only accept desired Intents, which
can limit usability. We believe that developers will trust Intent input by
default, allowing malicious input to potentially crash or abuse the application.

\paragraph{Proposal} ~\\
We plan to build a static analysis tool that will look for vulnerabilities in
Android apps that result from bad Intent processing. This analysis tool will
look at application use of the three Intent resolution classes, Activity,
BroadcastReceiver, and Service, that are not protected by system or signature
privileges. Intent input that is directly used by these classes without checking
or sanitization will be considered a potential vulnerability. We will test our
tool on apps found on the \href{https://play.google.com/store/apps}{Google Play}
store and use a dissasembly tool such as Dedexer \cite{dedexer} or
baksmali \cite{baksmali} to evaluate the application code. We will look for
potential vulnerabilities using the proposed static analysis tool and further
examine applications that use unchecked Intent input to access sensitive data or
privileges.\\

Using our static analysis tool, we plan on finding many interesting exploits
which provide access to sensitive data or privilege escalation. We will then
implement several of these exploits in a working, malicious application which we
will demo for the class.

\bibliography{proposal}
\end{document}
